\documentclass[11pt] {article}
\usepackage[margin=1in]{geometry} %one inch margins
\renewcommand{\baselinestretch}{2} %double space, safe for fancy headers

\begin{document}

RNA-sequencing (RNA-seq) uses next-generation sequencing (NGS) to estimate the quantity of RNA in biological samples at given timepoints. In recent years, decreasing cost and increasing throughput has rendered RNA-seq an attractive alternative to transcriptome profiling. Prior to RNA-seq, gene expression studies were performed with microarray techniques, which required prior knowledge of reference sequences. RNA-seq does not have this limitation, and has enabled a new range of applications such as transcriptome de novo assembly (Robertson et al., 2010) and detection of alternative splicing processes (Anders, Reyes, and Huber, 2012; Pan et al., 2008). Coupled with its high resolution and sensitivity, RNA-seq will likely revolutionize our understanding of the intricacies of eukaryotic transcriptomes (Wang, Gerstein, and Snyder, 2009; Zhao et al., 2014).

RNA-seq data is multivariate data, and its basic form is a matrix containing mapped read counts for \textit{n} rows of genes and \textit{p} columns of samples. These mapped read counts provide estimations of the gene expression levels across samples. Researchers typically conduct RNA-seq studies to identify differentially expressed genes (DEGs) between treatment groups. In most popular RNA-seq analysis packages, this objective is approached with models, such as the negative binomial model (Anders and Huber, 2010; Trapnell et al., 2013; Trapnell et al., 2012; Robinson et al., 2010) and linear regression models (Law et al., 2014).

Initially, it was widely claimed that RNA-seq produced unbiased data that did not require sophisticated normalization (Wang et al., 2009; Morin et al., 2008; Marioni et al., 2008). However, numerous studies have since revealed that RNA-seq data is replete with biases and that accurate detection of DEGs is not a negligible task. Problems that complicate the analysis of RNA-seq data include nucleotide and read-position biases (Hansen et al., 2010), biases related to gene lengths and sequencing depths (Oshlack, Robinson, and Young, 2010; Robinson and Oshlack, 2010), biases introduced during library preparation (McIntyre et al., 2011), biases pertaining to the number of replications (Schurch et al., 2016), biases derived from overlapping sense-antisense transcripts and gene isoforms (Trapnell et al., 2013), and the confounding combination of technical and biological variability (Bullard et al., 2010).

In light of these complications, researchers should analyze RNA-seq data like they would any other biased multivariate data. Simply applying models to such data is problematic because models hold assumptions that they alone cannot call into question. Fortunately, data visualization enables researchers to see patterns and problems they may not otherwise detect with traditional modeling. As a result, the most effective approach to data analysis is to iterate between models and visuals, and enhance the appropriateness of applied models based on feedback from visuals (Shneiderman, 2002). With RNA-seq data, we primarily want to compare the variability between replicates and between treatment groups. This is visually best achieved by drawing the mapped read count distributions across all genes and samples. Unfortunately, the few plotting tools offered in popular RNA-seq packages do not allow users to effectively view their data in this manner.

In this paper, we strive to remedy this problem by publishing new and effective RNA-seq plotting tools. We use real RNA-seq data to show that our tools can detect normalization problems, DEG designation problems, and common errors in the analysis pipeline. We also show that our tools can identify genes of interest that cannot otherwise be obtained by models. We emphasize that interactive graphics should be an indisposable component of modern RNA-seq analysis: Researchers should be able to quickly flip through plots of genes that appear promising or problematic, and link between plots to swiftly obtain various perspectives of their data. Here, we do not propose that users drastically change their approach to RNA-seq analysis. Instead, we propose that users simply modify their approach to RNA-seq analysis by assessing the sensibility of their models with multivariate graphical tools, namely with parallel coordinate plots, scatterplot matrices, and replicate point plots.

\end{document}